\documentclass{article}
%%%%%%%%%%%%%%%%%%%%%%%%%%%%% Define Article %%%%%%%%%%%%%%%%%%%%%%%%%%%%%%%%%%
%%%%%%%%%%%%%%%%%%%%%%%%%%%%%%%%%%%%%%%%%%%%%%%%%%%%%%%%%%%%%%%%%%%%%%%%%%%%%%%

%%%%%%%%%%%%%%%%%%%%%%%%%%%%% Using Packages %%%%%%%%%%%%%%%%%%%%%%%%%%%%%%%%%%
\usepackage{float}
\usepackage[letterpaper,portrait]{geometry}
\usepackage{graphicx}
\usepackage{anysize}
\usepackage{lipsum}
\usepackage{amsmath,amssymb,amsthm}
\usepackage[utf8]{inputenc}
\usepackage{multirow}
\usepackage{csquotes}
\usepackage[spanish]{babel}
\usepackage{apacite}
\usepackage{multicol}
\usepackage{parskip}
\usepackage{setspace}
\usepackage{empheq}
\usepackage{mdframed}
\usepackage{booktabs}
\usepackage{lipsum}
\usepackage{graphicx}
\usepackage{color}
\usepackage{psfrag}
\usepackage{pgfplots}
\usepackage{bm}
\usepackage{tocloft}

%%%%%%%%%%%%%%%%%%%%%%%%%%%%%%%%%%%%%%%%%%%%%%%%%%%%%%%%%%%%%%%%%%%%%%%%%%%%%%%

% Other Settings

%%%%%%%%%%%%%%%%%%%%%%%%%% Page Setting %%%%%%%%%%%%%%%%%%%%%%%%%%%%%%%%%%%%%%%
\geometry{letterpaper, margin=2.54cm}

%%%%%%%%%%%%%%%%%%%%%%%%%% Define some useful colors %%%%%%%%%%%%%%%%%%%%%%%%%%
\definecolor{ocre}{RGB}{243,102,25}
\definecolor{mygray}{RGB}{243,243,244}
\definecolor{deepGreen}{RGB}{26,111,0}
\definecolor{shallowGreen}{RGB}{235,255,255}
\definecolor{deepBlue}{RGB}{61,124,222}
\definecolor{shallowBlue}{RGB}{235,249,255}
%%%%%%%%%%%%%%%%%%%%%%%%%%%%%%%%%%%%%%%%%%%%%%%%%%%%%%%%%%%%%%%%%%%%%%%%%%%%%%%

%%%%%%%%%%%%%%%%%%%%%%%%%% Define an orangebox command %%%%%%%%%%%%%%%%%%%%%%%%
\newcommand\orangebox[1]{\fcolorbox{ocre}{mygray}{\hspace{1em}#1\hspace{1em}}}
%%%%%%%%%%%%%%%%%%%%%%%%%%%%%%%%%%%%%%%%%%%%%%%%%%%%%%%%%%%%%%%%%%%%%%%%%%%%%%%

%%%%%%%%%%%%%%%%%%%%%%%%%%%% English Environments %%%%%%%%%%%%%%%%%%%%%%%%%%%%%
\newtheoremstyle{mytheoremstyle}{3pt}{3pt}{\normalfont}{0cm}{\rmfamily\bfseries}{}{1em}{{\color{black}\thmname{#1}~\thmnumber{#2}}\thmnote{\,--\,#3}}
\newtheoremstyle{myproblemstyle}{3pt}{3pt}{\normalfont}{0cm}{\rmfamily\bfseries}{}{1em}{{\color{black}\thmname{#1}~\thmnumber{#2}}\thmnote{\,--\,#3}}
\theoremstyle{mytheoremstyle}
\newmdtheoremenv[linewidth=1pt,backgroundcolor=shallowGreen,linecolor=deepGreen,leftmargin=0pt,innerleftmargin=20pt,innerrightmargin=20pt,]{theorem}{Theorem}[section]
\theoremstyle{mytheoremstyle}
\newmdtheoremenv[linewidth=1pt,backgroundcolor=shallowBlue,linecolor=deepBlue,leftmargin=0pt,innerleftmargin=20pt,innerrightmargin=20pt,]{definition}{Definition}[section]
\theoremstyle{myproblemstyle}
\newmdtheoremenv[linecolor=black,leftmargin=0pt,innerleftmargin=10pt,innerrightmargin=10pt,]{problem}{Problem}[section]
%%%%%%%%%%%%%%%%%%%%%%%%%%%%%%%%%%%%%%%%%%%%%%%%%%%%%%%%%%%%%%%%%%%%%%%%%%%%%%%

%%%%%%%%%%%%%%%%%%%%%%%%%%%%%%% Plotting Settings %%%%%%%%%%%%%%%%%%%%%%%%%%%%%
\usepgfplotslibrary{colorbrewer}
\pgfplotsset{width=8cm,compat=1.9}
%%%%%%%%%%%%%%%%%%%%%%%%%%%%%%%%%%%%%%%%%%%%%%%%%%%%%%%%%%%%%%%%%%%%%%%%%%%%%%%

%%%%%%%%%%%%%%%%%%%%%%%%%%%%%%% Title & Author %%%%%%%%%%%%%%%%%%%%%%%%%%%%%%%%
\author{Gustavo Vergara}
%%%%%%%%%%%%%%%%%%%%%%%%%%%%%%%%%%%%%%%%%%%%%%%%%%%%%%%%%%%%%%%%%%%%%%%%%%%%%%%


\begin{document}
\pgfplotsset{compat=1.18}
\setstretch{2}

\begin{titlepage}
    \centering
    \vspace{2.5cm}
    {\scshape \Large Elaboración de historias de usuario del proyecto - GA2-220501093-AA1-EV03 \par}
    \vspace{5cm}
    \textbf\large\scshape{\par}
         \vspace{0.5cm}
         
    {\Large Vergara Pareja Gustavo\par}
    \vspace{5cm}
    {\scshape\Large Jovanna Herazo Daza\par}
    \vspace{0.3cm}
    {\scshape\Large Tecnología en Análisis y Desarrollo de Software \par}
    \vspace{0.3cm}
    {\scshape\Large SENA - Centro Agropecuario Regional Cauca\par}
    \vspace{0.3cm}
    {\Large \today \par}
    \end{titlepage}

\begin{flushleft}
    \large \textbf{EVIDENCIA A SOLUCIONAR}\\
    \vspace{0.1cm}
    \section*{Elaboración de historias de usuario del proyecto. GA2-220501093-AA1-EV03}
    
    Ahora, es momento de elaborar las historias de uso del proyecto, de acuerdo con los procesos ya elaborados, y por este motivo, se deben tener en cuenta los siguientes aspectos:
    \vspace{0.3cm}
    \newline Se deben seguir las normas básicas de presentación de un documento escrito; es decir, el documento debe tener como mínimo una portada, introducción, alcance, lista de requerimientos y versión del documento. Los requerimientos serán redactados usando el modelo IEEE830 y también el modelo de descripción de requisitos por medio de historias de usuario.
    \newline Respecto a lista de requerimientos, el aprendiz deberá agregar una sección donde se describa cada requisito usando los siguientes elementos del estándar IEEE830:
\begin{itemize}
    \item Perspectiva del producto
    \item Funciones del producto
    \item Características de los usuarios
    \item Restricciones
    \item Requisitos funcionales (formato de casos de uso)
    \item Requisitos no funcionales
\end{itemize}   
Respecto a la lista de requerimientos, el aprendiz deberá agregar una sección donde se describa cada requisito usando la estructura de historias de usuario, con los siguientes elementos por historia: 
\begin{itemize}
    \item Número de historia (priorizada)
    \item Nombre de la historia
    \item Usuario
    \item Descripción de la historia de usuario
    \item Observaciones
    \item Criterios de aceptación
\end{itemize}    
    \end{flushleft}
    \newpage
    \tableofcontents
    \newpage
\section{Introducción}

El presente documento describe los requisitos del sistema de gestión de citas médicas para la Clínica Regional de Montelíbano. Los requisitos se han elaborado utilizando el modelo IEEE830 y también el modelo de descripción de requisitos por medio de historias de usuario.

\section{Alcance}

El sistema de gestión de citas médicas debe permitir a los pacientes y al personal médico programar, cancelar y reprogramar citas médicas. El sistema también debe permitir al personal médico enviar recordatorios de citas a los pacientes.

\section{Requisitos}

\subsection{Perspectiva del producto}
\begin{itemize}
\item El sistema de gestión de citas médicas debe ser una herramienta útil y eficiente para los pacientes y el personal médico. 
\item El sistema debe ser fácil de usar y debe cumplir con los requisitos de la clínica.
\end{itemize}


\subsection{Funciones del producto}
El sistema de gestión de citas médicas debe permitir a los pacientes realizar las siguientes funciones:
\begin{itemize}
\item Registrarse en el sistema
\item Programar citas médicas
\item Cancelar citas médicas
\item Reprogramar citas médicas
\end{itemize}
El sistema de gestión de citas médicas debe permitir al personal médico realizar las siguientes funciones:
\begin{itemize}
\item Enviar recordatorios de citas a los pacientes
\item Administrar el calendario de citas
\item Reprogramar citas médicas
\end{itemize}

Los usuarios del sistema de gestión de citas médicas son los siguientes:
\begin{enumerate}
\item Pacientes
\item Personal médico
\end{enumerate}
El sistema de gestión de citas médicas debe cumplir con las siguientes restricciones:
\begin{itemize}
\item El sistema debe ser compatible con los navegadores web modernos.

\item El sistema debe ser accesible para personas con discapacidad.

\end{itemize}

\subsection{Requisitos funcionales}

Los requisitos funcionales del sistema de gestión de citas médicas se describen a continuación, utilizando el formato de casos de uso.

 \textbf{Caso de uso 1: Registrar paciente} 

Actor: Paciente

Propósito: Registrarse en el sistema de gestión de citas médicas.
\newpage
Flujo principal:
\begin{enumerate}
\item El paciente ingresa sus datos personales en el sistema.
\item El sistema verifica los datos ingresados.
\item Si los datos son correctos, el sistema crea una cuenta para el paciente.
\item El sistema muestra un mensaje de éxito.

\end{enumerate}

\textbf{Historia de usuario 1: } 
Como paciente, quiero poder registrarme en el sistema para poder programar citas médicas.

Prioridad: Alta

Usuario: Paciente

Puntos estimados de esfuerzo: 5

\textbf{Descripción:} 

El paciente debe poder ingresar sus datos personales, como nombre, apellido, correo electrónico, teléfono y fecha de nacimiento. El sistema debe verificar que los datos sean correctos antes de crear una cuenta para el paciente.

\textbf {Observaciones:}

\begin{itemize}
    \item El paciente debe poder recibir un correo electrónico de confirmación después de registrarse.
    \item El paciente debe poder elegir una contraseña para su cuenta.
\end{itemize} 
\textbf{Criterios de aceptación:}
\begin{itemize}
    \item El paciente debe poder ingresar sus datos personales correctamente.
    \item El sistema debe crear una cuenta para el paciente.
    \item El paciente debe recibir un correo electrónico de confirmación.
\end{itemize}  
\newpage 
\textbf{Caso de uso 2: Programar cita} \newline \
Actor: Paciente

Propósito: Programar una cita médica.

Flujo principal:
\begin{enumerate}
    \item El paciente selecciona un médico y una fecha y hora disponibles.
    \item El sistema muestra un resumen de la cita programada.
    \item El paciente confirma la cita.
    \item El sistema registra la cita en el calendario.
    \item El sistema muestra un mensaje de éxito.
    \end{enumerate}

    \textbf{Historia de usuario 2: }  
    Como paciente, quiero poder programar una cita médica con el médico que elija.

Prioridad: Alta

Usuario: Paciente

Puntos estimados de esfuerzo: 10

\textbf{Descripción:} 

El paciente debe poder seleccionar un médico y una fecha y hora disponibles para programar una cita. El sistema debe mostrar un resumen de la cita programada antes de que el paciente la confirme.

\textbf{ Observaciones:}
\begin{itemize}
    \item El paciente debe poder ver la disponibilidad de los médicos en tiempo real.
    \item El sistema debe verificar que la fecha y hora seleccionadas estén disponibles.
\end{itemize} 
\textbf{Criterios de aceptación:}
\begin{itemize}
    \item El paciente debe poder seleccionar un médico y una fecha y hora disponibles.
    \item El sistema debe mostrar un resumen de la cita programada.
    \item El paciente debe poder confirmar la cita.
\end{itemize}  

\textbf{Caso de uso 3: Cancelar cita} \newline \
Actor: Paciente

Propósito: Cancelar una cita médica programada.

Flujo principal:
\begin{enumerate} 
    \item El paciente selecciona una cita programada.
    \item El sistema muestra un mensaje de confirmación.
    \item El paciente confirma la cancelación.
    \item El sistema elimina la cita del calendario.
    \item El sistema muestra un mensaje de éxito.
    \end{enumerate}  



\textbf{Historia de usuario 3:}  
Como paciente, quiero poder cancelar una cita médica que ya he programado.

Prioridad: Alta

Usuario: Paciente

Puntos estimados de esfuerzo: 5

\textbf{Descripción:} 

El paciente debe poder seleccionar una cita programada para cancelarla. El sistema debe mostrar un mensaje de confirmación antes de que el paciente confirme la cancelación.

\textbf{Observaciones:} 
\begin{itemize}
    \item El paciente debe poder cancelar la cita sin penalización.
\end{itemize}  

\textbf{Criterios de aceptación:} 
\begin{itemize}
    \item El paciente debe poder seleccionar una cita programada.
    \item El sistema debe mostrar un mensaje de confirmación.
    \item El paciente debe poder confirmar la cancelación.
    \item El sistema debe eliminar la cita del calendario.
\end{itemize}  

\textbf{Caso de uso 4: Reprogramar cita} 

Actor: Paciente

Propósito: Reprogramar una cita médica programada.

Flujo principal:
\begin{enumerate} 
    \item El paciente selecciona una cita programada.
    \item El sistema muestra un calendario con las fechas y horas disponibles.
    \item El paciente selecciona una nueva fecha y hora.
    \item El sistema muestra un resumen de la cita reprogramada.
    \item El paciente confirma la reprogramación.
    \item El sistema actualiza la cita en el calendario.
    \item El sistema muestra un mensaje de éxito.
    \end{enumerate}  

\textbf{Historia de usuario 4:}  
Como paciente, quiero poder reprogramar una cita médica que ya he programado.

Prioridad: Alta

Usuario: Paciente

Puntos estimados de esfuerzo: 10

\textbf{Descripción:} 

El paciente debe poder seleccionar una cita programada para reprogramarla. El sistema debe mostrar un calendario con las fechas y horas disponibles antes de que el paciente confirme la reprogramación.

\textbf{Observaciones:} 

El paciente debe poder reprogramar la cita sin penalización.

\textbf{Criterios de aceptación:} 

El paciente debe poder seleccionar una cita programada.
El sistema debe mostrar un calendario con las fechas y horas disponibles.
El paciente debe poder seleccionar una nueva fecha y hora.
El sistema debe mostrar un resumen de la cita reprogramada.
El paciente debe poder confirmar la reprogramación.
El sistema debe actualizar la cita en el calendario.

\subsection{Requisitos no funcionales}
\textbf{Usabilidad:}
El sistema debe ser fácil de usar para pacientes y personal médico.
El sistema debe proporcionar una interfaz de usuario intuitiva.
\newline \textbf{Rendimiento:}
El sistema debe ser capaz de soportar un número elevado de usuarios simultáneos.
El sistema debe proporcionar un tiempo de respuesta rápido.
\newline \textbf{Disponibilidad:}
El sistema debe estar disponible 24/7.
\newline \textbf{Seguridad:}
El sistema debe proteger la privacidad de los datos de los pacientes.
\newline \textbf{Accesibilidad:}
El sistema debe ser accesible para personas con discapacidad.


\textbf{Observaciones:}

Los requisitos no funcionales se han clasificado según el estándar IEEE830.
Los requisitos no funcionales se han establecido de acuerdo con las necesidades de la Clínica Regional de Montelíbano.
\section{Conclusiones}

La elaboracion de historias de usuario es un documento valioso que ayudará a garantizar que el sistema de gestión de citas médicas sea efectivo y cumpla con las necesidades de la clínica sin dejar esacapar detalles.
%Encyclopædia Britannica. (n.d.). Computer programmers discuss software development. [Photograph]. Britannica ImageQuest. Retrieved October 19, 2023, from https://quest-eb-com.bdigital.sena.edu.co/images/132_1584681
%\bibliographystyle{apacite}
%\nocite{*}
%\bibliography{referenciados}

\end{document}